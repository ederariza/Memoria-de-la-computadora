\documentclass{article}
\usepackage[utf8]{inputenc}
\usepackage[spanish]{babel}
\usepackage{listings}
\usepackage{graphicx}
\graphicspath{ {images/} }
\usepackage{cite}
\usepackage[most]{tcolorbox}

\begin{document}
	
	\begin{titlepage}
		
		\begin{center}
			
			\vspace*{1cm}
			
			\Huge
			\textbf{Memoria de la computadora}
			
			\vspace{0.5cm}
			
			\LARGE
			Taller
			
			\vspace{0.5cm}
			
			\LARGE
			\textbf{Eder Alfonso Ariza Chamorro}\\
			Estudiante de Ingenieria de Telecomunicaciones\\
			
			\vspace*{1.5cm}
			
			\begin{figure}[h]
				\includegraphics[width=5cm]{Logo UdeA.png}
				\centering
				\label{fig:Logo UdeA}
			\end{figure}
			
			\vspace*{2cm}
			
			\Large
			Despartamento de Ingeniería Electrónica y Telecomunicaciones\\
			Universidad de Antioquia\\
			Medellín\\
			Septiembre 4 de 2020
			
		\end{center}
		
	\end{titlepage}
	
	
	\tableofcontents
	\newpage
	\section{Sección de preguntas}\label{contenido}
	
	\subsection{Defina que es la memoria del computador}
	
	\tcbset{colback=red!5!white,fonttitle=\bfseries}
	
	\begin{tcolorbox}[colupper=red!75!black]
		\textbf{Memoria}
		\tcblower
		Dispositivos de almacenamiento temporal y alta velocidad de acceso. Cumple un papel fundamental en el funcionamiento de una computadora.
	\end{tcolorbox}
	
	\subsection{Mencione los tipos de memoria que conoce y haga una pequeña descripción de cada tipo}
	
	\begin{tcolorbox}[colupper=red!75!black]
		\textbf{1. Memoria RAM}
		\tcblower
	\end{tcolorbox}
	\begin{tcolorbox}[colupper=red!75!black]
		\textbf{2. Memoria ROM}
		\tcblower
	\end{tcolorbox}
	\begin{tcolorbox}[colupper=red!75!black]
		\textbf{3. Memoria Cache}
		\tcblower
	\end{tcolorbox}
	\begin{tcolorbox}[colupper=red!75!black]
		\textbf{5. Memoria RAM estatica (SRAM)}
		\tcblower
	\end{tcolorbox}
	\begin{tcolorbox}[colupper=red!75!black]
		\textbf{6. Memoria Flash}
		\tcblower
	\end{tcolorbox}
	\begin{tcolorbox}[colupper=red!75!black]
		\textbf{7. Memoria Virtua}
		\tcblower
	\end{tcolorbox}
	\begin{tcolorbox}[colupper=red!75!black]
		\textbf{	8. Memoria de video (VRAM)}
		\tcblower
	\end{tcolorbox}
	\subsection{Describa la manera como se gestiona la memoria en un computador}
	
	\begin{tcolorbox}[colupper=red!75!black]
		\textbf{}
		\tcblower
		
	\end{tcolorbox}
	
	\subsection{¿Qué hace que una memoria sea más rápida que otra? ¿Por qué esto es importante?}
	
	
\end{document}
