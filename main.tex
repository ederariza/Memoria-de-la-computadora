\documentclass{article}
\usepackage[utf8]{inputenc}
\usepackage[spanish]{babel}
\usepackage{graphicx}
\graphicspath{ {images/} }
\usepackage{cite}
\usepackage[most]{tcolorbox}

\begin{document}

\begin{titlepage}
	
    \begin{center}
	
		\vspace*{1cm}
	
		\Huge
		\textbf{Memoria de la computadora}
	
		\vspace{0.5cm}
	
		\LARGE
		Taller
	
		\vspace{0.5cm}
	
		\LARGE
		\textbf{Eder Alfonso Ariza Chamorro}\\
		Estudiante de Ingenieria de Telecomunicaciones\\
	
		\vspace*{1.5cm}
	
		\begin{figure}[h]
			\includegraphics[width=5cm]{Logo UdeA.png}
			\centering
			\label{fig:Logo UdeA}
		\end{figure}
	
		\vspace*{2cm}
	
		\Large
		Despartamento de Ingeniería Electrónica y Telecomunicaciones\\
		Universidad de Antioquia\\
		Medellín\\
		Septiembre 4 de 2020
	
	\end{center}

\end{titlepage}

\tableofcontents
\newpage


\section{Introduccion}\label{intro}
En el presente trabajo aprenderemos la importacia acerca del funcionamiento, conformacion y clasifiacion de las memorias del computador.\\

Es un trabajo que se ira profundizando y mejorando a travez del transcruso del semestre.

\section{Sección de preguntas}\label{contenido}

\subsection{Defina que es la memoria del computador}

\tcbset{colback=red!5!white,fonttitle=\bfseries}

\begin{tcolorbox}[colupper=red!75!black]
	\textbf{Memoria}
	\tcblower
	Dispositivos de almacenamiento temporal y alta velocidad de acceso a la informacion requerida por el microprocesador. Cumple un papel fundamental en el funcionamiento de una computadora.
\end{tcolorbox}

\subsection{Mencione los tipos de memoria que conoce y haga una pequeña descripción de cada tipo}

\begin{tcolorbox}[colupper=red!75!black]
	\textbf{Memoria Cache}
	\tcblower
	La memoria cache, es una memoria de acceso rapido. Su principal funcion es servir de apoyo al microprocesador, puesto que puede trabajar a altas velocidades. Tiene diferentes niveles: L1, L2, L3. La mas rapida y cercana al microproceador es el nivel L1, tambien es la que tiene menor capacidad. Seguido del nivel L2 un poco mas lento que L1, pero con mayor capacidad. Y finalmente el nivel L3 el cual es mas lento que L1 y que L2. Pero, con mayo capacidad que las dos anteriores y mucho mas rapido que la memoria RAM.
\end{tcolorbox}

\begin{tcolorbox}[colupper=red!75!black]
	\textbf{Memoria RAM}
	\tcblower
	Es la memoria peincipal de la computadora, y se encarga de almacenar la informacion durante la ejecuccion del sofware. Alamacena los datos de forma temporal, es por esto que se le conoce como una memoria volatil. Puesto que, al cesar el flujo de corriente los datos no quedan almacenados.Ademaas, es aleatoria porque puede aceder a los datos de forma desordenada.
\end{tcolorbox}

\begin{tcolorbox}[colupper=red!75!black]
	\textbf{Memoria Virtual}
	\tcblower
	Porcion del dico duro dedicado temporalmente para complementar la memoria RAM, en caso de que esta este por agotarse
\end{tcolorbox}

\begin{tcolorbox}[colupper=red!75!black]
	\textbf{Disco duro}
	\tcblower
	Es un tipo de dispositivo de memoria que permite almacenar grandes cantidades de informacion, incluso en auscencia de corriente electrica.
\end{tcolorbox}

\subsection{Describa la manera como se gestiona la memoria en un computador}
\begin{tcolorbox}[colupper=red!75!black]
	\textbf{Controlador de memoria}
	\tcblower
	El controlador de memoria se encarga de gestionar cada tarea de la memoria, comunicando las instrucciones del microprocesador, interviniendo en cada transferencia de información desde y hacia la memoria, y estableciendo el ritmo o la velocidad con que se realizan las operaciones a través de su reloj que marca millones de ciclos por segundo (medidos en Mhz o Megahertz).
	\vspace{0.1cm}
	Dicho controlador de memoria puede encontrarse en uno de los siguientes dos lugares:\\
	1. En un chip ubicado en la placa madre entre los módulos de memoria y la CPU o el microprocesador, y que se denomina Northbridge o MCH (Memory Controller Hub - Centro de Control de Memoria).\\
	2. En sistemas más modernos se encuentra incorporado dentro del microprocesador.
\end{tcolorbox}

\subsection{¿Qué hace que una memoria sea más rápida que otra? ¿Por qué esto es importante?}
Para intentar dar respuesta a esta pregunta debemos tener claros los siguientes conceptos:\\
\begin{tcolorbox}[colupper=red!75!black]
	\textbf{Latencia}
	\tcblower
	Medida que se suele utilizar para medir el nivel de eficiencia de un módulo de memoria es
	la latencia, la cual está relacionada con el proceso electrónico de lectura y escritura de los datos en la memoria SDRAM. en otras palabras la latencia es la cantidad de tiempo que se tarda en obtener de la memoria cada bit de información. En otras palabras, podria decirse que es el tiempo que pasa desde que el controlador de memoria realiza una peticion y obtiene una respuesta.
\end{tcolorbox}

\begin{tcolorbox}[colupper=red!75!black]
	\textbf{Frecuencia}
	\tcblower
	Ritmo de trabajo o velocidad con la que se comunica la memoria con el bus de control.
\end{tcolorbox}
Lo anterior es importante a tener en cuenta, puesto que puede afectar el rendimiento de nuestra memoria.

\section{Citación}
Documento basado en el artículo: Como funciona la memoria de una computadora por \textbf{Augusto Salazar} \cite{Salazar}.

\bibliographystyle{IEEEtran}
\bibliography{references}

\end{document}
